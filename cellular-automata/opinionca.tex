%%%%%%%%%%%%%%%%%%%%
%%%%%%%%%%%%%%%%%%%%
%%
%% Andrea Tino - 2019
%% Programming + Science
%% Opinion model
%%
%%%%%%%%%%%%%%%%%%%%
%%%%%%%%%%%%%%%%%%%%

\section{Developing a CA to describe people's opinion change}
\label{sec:opinionca}

In section \ref{sec:simpleca}, we have built our first automaton: Conway's Game of Life.
That CA is a great start because it has many interesting configurations and evolutions;
however, now, we want to move forward and develop another, different, automaton.
A CA is a mathematical model; other than being a very fun thing to play with, it is a
tool that can be used to study our reality from a theoretical perspective. Like any other
model, it is capable of simplifying our universe so that we can study specific things
about a natural phenomenon. CGL was just an automaton we built without a specific goal
in mind, we just wanted to play with automata.
For the next stage, we want to build a CA that can help us reach a distinct objective:
illustrating opinion change among the members of a society.

\subsection{Working out the model}
We want to use a scientific approach to solve a problem. So, 
before going straight to coding, we need to:

\begin{enumerate}
\item Decide what natural phenomenon we want to describe and control. 
We basically want to answer the question: what is the problem we want to solve?
\item Define the mathematical model to reach that objective. We 
basically need to translate
our problem into mathematical terms. This step is called: \textit{modelization}.
\item Create a computer simulation by translating into code the mathematical
model we created, this stage is called: \textit{implementation}.
\end{enumerate}

This way of doing things is called: \textit{scientific approach} and is at the very core
of what scientists, mathematicians and engineers do every day. So let's start
with the first step: what problem do we want to solve?

\begin{proposition}[Problem definition]
\label{prop:opinionproblem}
We want to
\end{proposition}

fg
