%%%%%%%%%%%%%%%%%%%%
%%%%%%%%%%%%%%%%%%%%
%%
%% Andrea Tino - 2019
%% Programming + Science
%% Model
%%
%%%%%%%%%%%%%%%%%%%%
%%%%%%%%%%%%%%%%%%%%

\section{Defining the model}
\label{sec:model}

Following the typical approaches in Computational Linguistics, we need to formalise the target language.
A valid sign is therefore represented by an ordered 2-dimension tuple:

\begin{definition}[Full sign token] \label{def:fst}
Let $\mathcal{S}$ be the set of full sign tokens, one token $s \in \mathcal{S}$ will be indicated as:
$s = (\Sigma, \sigma)$ where $\Sigma \in S$ represents the dominant sign (signed by the subject with his
dominant hand), and $\sigma \in S$ represents the corresponding non-dominant sign in $s$.
\end{definition}

Note how definition \ref{def:fst} does not make assumptions on left-hand or right-hand. The only distinction
is made on the dominancy of the single signs forming the full sign.

\begin{definition}[Full sign token flow]
Let $\mathcal{S}$ be the set of full sign tokens, we indicate with 
$\mathcal{S}^\ast = \mathcal{S} \cup \mathcal{S}^2 \cup \dots \cup \mathcal{S}^\infty$ the set of all possible
strings made up of tokens from $\mathcal{S}$. A token-string, or token-flow, is denoted by expression:
$\Phi = s_1 s_2 \dots s_n \in \mathcal{S}^\ast$, where $s_i \in \mathcal{S}$ for $i = 1 \dots n$, and 
$n = |\Phi|$.
\end{definition}

In the context of our effort, we will not handle strings of full tokens $\Phi \in \mathcal{S}^\ast$ as we will
be missing, for each token $s = (\Sigma, \sigma) \in \mathcal{S}$, the information on $\sigma$.
