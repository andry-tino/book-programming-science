%%%%%%%%%%%%%%%%%%%%
%%%%%%%%%%%%%%%%%%%%
%%
%% Andrea Tino - 2019
%% Programming + Science
%% Opinion model
%%
%%%%%%%%%%%%%%%%%%%%
%%%%%%%%%%%%%%%%%%%%

\section{Building our first CA: Conway's Game of Life}
\label{sec:simpleca}

We want to create our first CA in code so that we can display the cells and see them changing
state. In this section, we will build the basic architecture of a CA that can be used to build any
CA in future. For this, we are going to use the latest web technologies to create web sites and
web applications in the browser: Javascript, HTML and CSS.

\subsection{Creating the basic project structure}
In our PC, let's create a directory (anywhere you want, on your Desktop maybe?) and give it
a cool name like: \texttt{cellautom}. Inside this new directory, do the following:

\begin{enumerate}
\item Create a file and name it: \texttt{index.html}.
\item Create a file and name it: \texttt{ca.js}.
\item Create one last file and name it: \texttt{style.css}.
\end{enumerate}

These three files represent the basic organization of our visual CA we are going to develop.

% Use the \index{} command to code your index words
%
% For tables use
%
\begin{table}[!t]
\caption{Please write your table caption here}
\label{tab:1}       % Give a unique label
%
% Follow this input for your own table layout
%
\begin{tabular}{p{2cm}p{2.4cm}p{2cm}p{4.9cm}}
\hline\noalign{\smallskip}
Classes & Subclass & Length & Action Mechanism  \\
\noalign{\smallskip}\svhline\noalign{\smallskip}
Translation & mRNA$^a$  & 22 (19--25) & Translation repression, mRNA cleavage\\
Translation & mRNA cleavage & 21 & mRNA cleavage\\
Translation & mRNA  & 21--22 & mRNA cleavage\\
Translation & mRNA  & 24--26 & Histone and DNA Modification\\
\noalign{\smallskip}\hline\noalign{\smallskip}
\end{tabular}
\end{table}
%

as