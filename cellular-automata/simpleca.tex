%%%%%%%%%%%%%%%%%%%%
%%%%%%%%%%%%%%%%%%%%
%%
%% Andrea Tino - 2019
%% Programming + Science
%% Opinion model
%%
%%%%%%%%%%%%%%%%%%%%
%%%%%%%%%%%%%%%%%%%%

\section{Building our first CA: Conway's Game of Life}
\label{sec:simpleca}

We want to create our first CA in code so that we can display the cells and see them changing
state. In this section, we will build the basic architecture of a CA that can be used to build any
CA in future. For this, we are going to use the latest web technologies to create web sites and
web applications in the browser: Javascript, HTML and CSS.

\subsection{Creating the basic project structure}
In our PC, let's create a directory (anywhere you want, on your Desktop maybe?) and give it
a cool name like: \texttt{cellautom}. Inside this new directory, do the following:

\begin{enumerate}
\item Create a file and name it: \texttt{index.html}.
\item Create a file and name it: \texttt{ca.js}.
\item Create one last file and name it: \texttt{style.css}.
\end{enumerate}

These three files represent the basic organization of our visual CA we are going to develop.
Table \ref{tab:files} offers a good overview of what they are needed for.

%
% Table
%
\begin{table}[!t]
\centering
\caption{Please write your table caption here}
\label{tab:files}
%
% Follow this input for your own table layout
%
\begin{tabular}{p{0.2\textwidth}p{0.2\textwidth}p{0.5\textwidth}}
\hline\noalign{\smallskip}
File & Type & Description \\
\noalign{\smallskip}\svhline\noalign{\smallskip}
\texttt{index.html} & Web page & This is the web pae that will display the CA and its evolution.\\
\texttt{ca.js} & Javascript code file  & This file will contain the J
    avascript code that will make the CA appear and evolve.\\
\texttt{style.css} & CSS Stylesheet  & The stylesheet we will use to apply colors, fonts 
    and make our CA beautiful.\\
\noalign{\smallskip}\hline\noalign{\smallskip}
\end{tabular}
\end{table}
%

% Figure
%
\begin{figure}[b]
\sidecaption
% tikz diagram
%
% Forest Diagram
%

\begin{forest}
    for tree={
      font=\ttfamily,
      grow'=0,
      child anchor=west,
      parent anchor=south,
      anchor=west,
      calign=first,
      inner xsep=7pt,
      edge path={
        \noexpand\path [draw, \forestoption{edge}]
        (!u.south west) +(7.5pt,0) |- (.child anchor) pic {folder} \forestoption{edge label};
      },
      % style for file node 
      file/.style={edge path={\noexpand\path [draw, \forestoption{edge}]
        (!u.south west) +(7.5pt,0) |- (.child anchor) \forestoption{edge label};},
        inner xsep=2pt,font=\small\ttfamily
                   },
      before typesetting nodes={
        if n=1
          {insert before={[,phantom]}}
          {}
      },
      fit=band,
      before computing xy={l=15pt},
    }  
  [your-computer
    [cellautom
      [index.html,file
      ]
      [ca.js,file
      ]
      [style.css,file
      ]
    ]
  ]
\end{forest}

%
% If not, use
%\picplace{5cm}{2cm} % Give the correct figure height and width in cm
%
\caption{Your project folder should look like this.}
\label{fig:dirstruct1}
\end{figure}
%

\subsection{Defining the barebones}
Time to write some initial code to see something appearing on the page once we run in the browser.
Open file \texttt{index.html} with tyour favorite editor and input this code:

\begin{programcode}{index.html}
Write this code minding casing and spacing.
\begin{verbatim}
<!DOCTYPE html>
<html>
<head>
  <title>My Cellular Automaton</title>
</head>

<body>
  Hello world!
</body>
</html>
\end{verbatim}
\end{programcode}

Save the file and now try to open it in your browser.

\begin{tips}{A first glance at HTML}
The code we just wrote is read by the browser to create a graphical visualization. HTML is used to
create web pages. It is not a \textit{programming language} (it does not tell a computer what to do),
but a \textit{markup language} (it tells a computer what to display and paint on the screen).

A minimal HTML page is exactly the one we wrote. It is all based on the concept of \textit{tags}. The
first line \texttt{<!DOCTYPE html>} is special and tells the browser that we are using the latest
version of HTML (you should always use this). Then a new tag \texttt{<html>} is opened and is
closed at the end of the file: \texttt{</html>}. An opening tag and a closing tag make a \textit{block}.
Blocks can contain other blocks.

The \texttt{<html>} block must contain, in order, two other blocks:
\texttt{<head>} and \texttt{<body>}. The first block contains the block for defining the title of the page
(this text is displayed on the browser's top bar). Everything inside \texttt{<head>} will not generate any 
graphics, it only contains information about the page. What's inside \texttt{body} is, on the other hand, 
painted (or, more technically speaking, rendered\footnote{The term \textit{render} is used to indicate
the complex set of operations that a program does in order to visualize something on the screen.})
inside the browser window. As you can see, we only have a piece of 
text\footnote{The \textit{Hello World} is, historically, the first thing one learns
to do when learning a new programming language, we had to respect tradition here.}, which
is in fact rendered on a blank, empty page.
\end{tips}

Of course we don't want to just display text, we want to render a full CA! So, in the same file, replace
that text.

\begin{programcode}{index.html (snippet)}
Remove \texttt{Hello world!} and insert a \texttt{<div>} block instead.
\begin{verbatim}
<body>
  <div id="ca"></div>
</body>
\end{verbatim}
\end{programcode}

A \texttt{<div>} block is used to group things. We are going to write some code that puts some graphics
inside it. Before leaving this file, we need to import inside it the other two files we have created.

\begin{programcode}{index.html (snippet)}
Place these new tags below block \texttt{<title>}.
\begin{verbatim}
<head>
  <title>My Cellular Automaton</title>
  <script src="./ca.js"></script>
</head>
\end{verbatim}
\end{programcode}

For now we are done with \texttt{index.html}, so let.
