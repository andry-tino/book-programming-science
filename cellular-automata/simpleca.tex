%%%%%%%%%%%%%%%%%%%%
%%%%%%%%%%%%%%%%%%%%
%%
%% Andrea Tino - 2019
%% Programming + Science
%% Opinion model
%%
%%%%%%%%%%%%%%%%%%%%
%%%%%%%%%%%%%%%%%%%%

\section{Building our first CA: Conway's Game of Life}
\label{sec:simpleca}

We want to create our first CA in code so that we can display the cells and see them changing
state. In this section, we will build the basic architecture of a CA that can be used to build any
CA in future. For this, we are going to use the latest web technologies to create web sites and
web applications in the browser: Javascript, HTML and CSS.

\subsection{Creating the basic project structure}
In our PC, let's create a directory (anywhere you want, on your Desktop maybe?) and give it
a cool name like: \texttt{cellautom}. Inside this new directory, do the following:

\begin{enumerate}
\item Create a file and name it: \texttt{index.html}.
\item Create a file and name it: \texttt{ca.js}.
\item Create one last file and name it: \texttt{style.css}.
\end{enumerate}

These three files represent the basic organization of our visual CA we are going to develop.
Table \ref{tab:files} offers a good overview of what they are needed for.

%
% Table
%
\begin{table}[!t]
\centering
\caption{Please write your table caption here}
\label{tab:files}
%
% Follow this input for your own table layout
%
\begin{tabular}{p{0.2\textwidth}p{0.2\textwidth}p{0.5\textwidth}}
\hline\noalign{\smallskip}
File & Type & Description \\
\noalign{\smallskip}\svhline\noalign{\smallskip}
\texttt{index.html} & Web page & This is the web pae that will display the CA and its evolution.\\
\texttt{ca.js} & Javascript code file  & This file will contain the J
    avascript code that will make the CA appear and evolve.\\
\texttt{style.css} & CSS Stylesheet  & The stylesheet we will use to apply colors, fonts 
    and make our CA beautiful.\\
\noalign{\smallskip}\hline\noalign{\smallskip}
\end{tabular}
\end{table}
%

\subsection{Defining the barebones}
Time to write some initial code to see something appearing on the page once we run in the browser.
Open file \texttt{index.html} with tyour favorite editor and input this code:

\begin{programcode}{Code: index.html}
Write this code minding casing and spacing.
\begin{verbatim}
<!DOCTYPE html>
<html>
<head>
  <title>My Cellular Automaton</title>
</head>

<body>
  Hello world!
</body>
</html>
\end{verbatim}
\end{programcode}

Save the file and now try to open it in your browser.

\begin{tips}{Coding tip: a first glance at HTML}
We have just written down oue first HTML snippet!
\end{tips}

df
