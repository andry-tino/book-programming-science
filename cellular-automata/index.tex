%
% Andrea Tino
% 2020
%
% =======================================================================
% Creative Commons Attribution-NonCommercial-ShareAlike 4.0 International
% Public License
% =======================================================================
%

\documentclass[twoside,symmetric,justified]{tufte-book}

\hypersetup{colorlinks}% uncomment this line if you prefer colored hyperlinks (e.g., for onscreen viewing)

%%
% Book metadata
\title{Studying\\populations using\\Cellular Automata}
\author[Andrea Tino]{Andrea Tino}
\publisher{An Open Source Book}

%%
% If they're installed, use Bergamo and Chantilly from www.fontsite.com.
% They're clones of Bembo and Gill Sans, respectively.
%\IfFileExists{bergamo.sty}{\usepackage[osf]{bergamo}}{}% Bembo
%\IfFileExists{chantill.sty}{\usepackage{chantill}}{}% Gill Sans

%\usepackage{microtype}

%%
% Just some sample text
\usepackage{lipsum}

%%
% For nicely typeset tabular material
\usepackage{booktabs}



%%
% TIKZ
\usepackage{tikz}
\usetikzlibrary{plotmarks}
\usetikzlibrary{patterns}
\usetikzlibrary{decorations.markings}
\usetikzlibrary{math}
\usetikzlibrary{matrix}
\usetikzlibrary{arrows,tikzmark,shadows,positioning}

%%
% Algorithms
\usepackage{algorithm}
\usepackage{algpseudocode}

%%
% Forest diagrams
\usepackage{forest}

%%
% Misc
\usepackage{xstring}

%%
% Listings
%\usepackage{xcolor}
%\definecolor{light-gray}{gray}{0.90}

\usepackage{listings}
\usepackage{lstlinebgrd}

\lstset{basicstyle=\ttfamily\footnotesize,breaklines=true}

%
% ECMAScript 2015 (ES6) definition
%

\lstdefinelanguage[ECMAScript2015]{JavaScript}[]{JavaScript}{
  morekeywords=[1]{await, async, case, catch, class, const, default, do,
    enum, export, extends, finally, from, implements, import, instanceof,
    let, static, super, switch, throw, try},
  morestring=[b]` % Interpolation strings.
}

%
% JavaScript version 1.1
%

\lstdefinelanguage{JavaScript}{
  morekeywords=[1]{break, continue, delete, else, for, function, if, in,
    new, return, this, typeof, var, void, while, with},
  % Literals, primitive types, and reference types.
  morekeywords=[2]{false, null, true, boolean, number, undefined,
    Array, Boolean, Date, Math, Number, String, Object, window, document},
  % Built-ins.
  morekeywords=[3]{eval, parseInt, parseFloat, escape, unescape},
  sensitive,
  morecomment=[s]{/*}{*/},
  morecomment=[l]//,
  morecomment=[s]{/**}{*/}, % JavaDoc style comments
  morestring=[b]',
  morestring=[b]"
}[keywords, comments, strings]

\lstalias[]{ES6}[ECMAScript2015]{JavaScript}

\lstdefinestyle{JavaScript}{
  language=JavaScript
}
\lstdefinestyle{JavaScript2}{
  language=JavaScript,
  basicstyle=\ttfamily\footnotesize\color{gray}
}
\lstdefinestyle{ES6}{
  language=ES6
}

%%%

\lstnewenvironment{codecss}{}{}

\lstnewenvironment{code}
{
\lstset{style=JavaScript}
}
{
}

\lstnewenvironment{codehtml}
{
\lstset{language=HTML}
}
{
}

\lstnewenvironment{codehtmlh1}[2]
{
\lstset{
        language=HTML,
        linebackgroundcolor={\ifnum\value{lstnumber}<#2\ifnum\value{lstnumber}>#1\color{light-gray}\fi\fi}}
}
{
}

\lstnewenvironment{codeh1}[2]
{
\lstset{
        style=JavaScript,
        linebackgroundcolor={\ifnum\value{lstnumber}<#2\ifnum\value{lstnumber}>#1\color{light-gray}\fi\fi}}
}
{
}

\lstnewenvironment{codeh2}[4]
{
\lstset{
        style=JavaScript,
        linebackgroundcolor={
        \ifnum \value{lstnumber}<#2
          \ifnum\value{lstnumber}>#1
            \color{light-gray}
          \fi
        \else
          \ifnum \value{lstnumber}<#4
            \ifnum \value{lstnumber}>#3
              \color{light-gray}
            \fi
          \fi
        \fi}}
}
{
}

%%
% Colors
\definecolor{folderbg}{RGB}{124,166,198}
\definecolor{folderborder}{RGB}{110,144,169}
\def\FolderSize{4pt}
\tikzset{
  folder/.pic={
    \filldraw[draw=folderborder,top color=folderbg!50,bottom color=folderbg]
      (-1.05*\FolderSize,0.2\FolderSize+5pt) rectangle ++(.75*\FolderSize,-0.2\FolderSize-5pt);  
    \filldraw[draw=folderborder,top color=folderbg!50,bottom color=folderbg]
      (-1.15*\FolderSize,-\FolderSize) rectangle (1.15*\FolderSize,\FolderSize);
  }
}

%%
% For graphics / images
\usepackage{graphicx}
\setkeys{Gin}{width=\linewidth,totalheight=\textheight,keepaspectratio}
\graphicspath{{graphics/}}

% The fancyvrb package lets us customize the formatting of verbatim
% environments.  We use a slightly smaller font.
\usepackage{fancyvrb}
\fvset{fontsize=\normalsize}

%%
% Prints argument within hanging parentheses (i.e., parentheses that take
% up no horizontal space).  Useful in tabular environments.
\newcommand{\hangp}[1]{\makebox[0pt][r]{(}#1\makebox[0pt][l]{)}}

%%
% Prints an asterisk that takes up no horizontal space.
% Useful in tabular environments.
\newcommand{\hangstar}{\makebox[0pt][l]{*}}

%%
% Prints a trailing space in a smart way.
\usepackage{xspace}

%%
% Math
\usepackage{amsmath}  % extended mathematics
\usepackage{amsfonts}  % extended mathematics
\usepackage{amssymb}  % extended mathematics
\usepackage{amsthm}

\newtheorem{theorem}{Theorem}
\newtheorem{corollary}{Corollary}
\newtheorem{example}{Example}
\newtheorem{problem}{Problem}
\newtheorem{proposition}{Proposition}
\newtheorem{lemma}[theorem]{Lemma}
\newtheorem{definition}{Definition}

\newcommand{\probref}[1]{\textbf{\ref{#1}} }
\newenvironment{sol}{\par\addvspace{6pt}\noindent\probref}{\par\addvspace{6pt}}

\newcommand{\E}{\mathrm{E}}
\newcommand{\median}{\mathrm{median}}
\newcommand{\Var}{\mathrm{Var}}
\newcommand{\EPE}{\mathrm{EPE}}
\newcommand{\argmin}{\mathop{\mathrm{argmin}}}

\definecolor{light-gray}{gray}{0.90}

%%
% Some shortcuts for Tufte's book titles.  The lowercase commands will
% produce the initials of the book title in italics.  The all-caps commands
% will print out the full title of the book in italics.
\newcommand{\vdqi}{\textit{VDQI}\xspace}
\newcommand{\ei}{\textit{EI}\xspace}
\newcommand{\ve}{\textit{VE}\xspace}
\newcommand{\be}{\textit{BE}\xspace}
\newcommand{\VDQI}{\textit{The Visual Display of Quantitative Information}\xspace}
\newcommand{\EI}{\textit{Envisioning Information}\xspace}
\newcommand{\VE}{\textit{Visual Explanations}\xspace}
\newcommand{\BE}{\textit{Beautiful Evidence}\xspace}

\newcommand{\TL}{Tufte-\LaTeX\xspace}

% Prints the month name (e.g., January) and the year (e.g., 2008)
\newcommand{\monthyear}{%
  \ifcase\month\or January\or February\or March\or April\or May\or June\or
  July\or August\or September\or October\or November\or
  December\fi\space\number\year
}


% Prints an epigraph and speaker in sans serif, all-caps type.
\newcommand{\openepigraph}[2]{%
  %\sffamily\fontsize{14}{16}\selectfont
  \begin{fullwidth}
  \sffamily\large
  \begin{doublespace}
  \noindent\allcaps{#1}\\% epigraph
  \noindent\allcaps{#2}% author
  \end{doublespace}
  \end{fullwidth}
}

% Inserts a blank page
\newcommand{\blankpage}{\newpage\hbox{}\thispagestyle{empty}\newpage}

\usepackage{units}

% Typesets the font size, leading, and measure in the form of 10/12x26 pc.
\newcommand{\measure}[3]{#1/#2$\times$\unit[#3]{pc}}

% Macros for typesetting the documentation
\newcommand{\hlred}[1]{\textcolor{Maroon}{#1}}% prints in red
\newcommand{\hangleft}[1]{\makebox[0pt][r]{#1}}
\newcommand{\hairsp}{\hspace{1pt}}% hair space
\newcommand{\hquad}{\hskip0.5em\relax}% half quad space
\newcommand{\TODO}{\textcolor{red}{\bf TODO!}\xspace}
\newcommand{\ie}{\textit{i.\hairsp{}e.}\xspace}
\newcommand{\eg}{\textit{e.\hairsp{}g.}\xspace}
\newcommand{\na}{\quad--}% used in tables for N/A cells
\providecommand{\XeLaTeX}{X\lower.5ex\hbox{\kern-0.15em\reflectbox{E}}\kern-0.1em\LaTeX}
\newcommand{\tXeLaTeX}{\XeLaTeX\index{XeLaTeX@\protect\XeLaTeX}}
% \index{\texttt{\textbackslash xyz}@\hangleft{\texttt{\textbackslash}}\texttt{xyz}}
\newcommand{\tuftebs}{\symbol{'134}}% a backslash in tt type in OT1/T1
\newcommand{\doccmdnoindex}[2][]{\texttt{\tuftebs#2}}% command name -- adds backslash automatically (and doesn't add cmd to the index)
\newcommand{\doccmddef}[2][]{%
  \hlred{\texttt{\tuftebs#2}}\label{cmd:#2}%
  \ifthenelse{\isempty{#1}}%
    {% add the command to the index
      \index{#2 command@\protect\hangleft{\texttt{\tuftebs}}\texttt{#2}}% command name
    }%
    {% add the command and package to the index
      \index{#2 command@\protect\hangleft{\texttt{\tuftebs}}\texttt{#2} (\texttt{#1} package)}% command name
      \index{#1 package@\texttt{#1} package}\index{packages!#1@\texttt{#1}}% package name
    }%
}% command name -- adds backslash automatically
\newcommand{\doccmd}[2][]{%
  \texttt{\tuftebs#2}%
  \ifthenelse{\isempty{#1}}%
    {% add the command to the index
      \index{#2 command@\protect\hangleft{\texttt{\tuftebs}}\texttt{#2}}% command name
    }%
    {% add the command and package to the index
      \index{#2 command@\protect\hangleft{\texttt{\tuftebs}}\texttt{#2} (\texttt{#1} package)}% command name
      \index{#1 package@\texttt{#1} package}\index{packages!#1@\texttt{#1}}% package name
    }%
}% command name -- adds backslash automatically
\newcommand{\docopt}[1]{\ensuremath{\langle}\textrm{\textit{#1}}\ensuremath{\rangle}}% optional command argument
\newcommand{\docarg}[1]{\textrm{\textit{#1}}}% (required) command argument
\newenvironment{docspec}{\begin{quotation}\ttfamily\parskip0pt\parindent0pt\ignorespaces}{\end{quotation}}% command specification environment
\newcommand{\docenv}[1]{\texttt{#1}\index{#1 environment@\texttt{#1} environment}\index{environments!#1@\texttt{#1}}}% environment name
\newcommand{\docenvdef}[1]{\hlred{\texttt{#1}}\label{env:#1}\index{#1 environment@\texttt{#1} environment}\index{environments!#1@\texttt{#1}}}% environment name
\newcommand{\docpkg}[1]{\texttt{#1}\index{#1 package@\texttt{#1} package}\index{packages!#1@\texttt{#1}}}% package name
\newcommand{\doccls}[1]{\texttt{#1}}% document class name
\newcommand{\docclsopt}[1]{\texttt{#1}\index{#1 class option@\texttt{#1} class option}\index{class options!#1@\texttt{#1}}}% document class option name
\newcommand{\docclsoptdef}[1]{\hlred{\texttt{#1}}\label{clsopt:#1}\index{#1 class option@\texttt{#1} class option}\index{class options!#1@\texttt{#1}}}% document class option name defined
\newcommand{\docmsg}[2]{\bigskip\begin{fullwidth}\noindent\ttfamily#1\end{fullwidth}\medskip\par\noindent#2}
\newcommand{\docfilehook}[2]{\texttt{#1}\index{file hooks!#2}\index{#1@\texttt{#1}}}
\newcommand{\doccounter}[1]{\texttt{#1}\index{#1 counter@\texttt{#1} counter}}

%%%%%%%%%%%%%%%%%%%%%
%%%%%%%%%%%%%%%%%%%%%
%%%%%%%%%%%%%%%%%%%%%

% Generates the index
\usepackage{makeidx}
\makeindex

\begin{document}

% Front matter
\frontmatter


% r.3 full title page
\maketitle


% v.4 copyright page
\newpage
\begin{fullwidth}
~\vfill
\thispagestyle{empty}
\setlength{\parindent}{0pt}
\setlength{\parskip}{\baselineskip}
Copyright \copyright\ \the\year\ \thanklessauthor

\par\smallcaps{Publishing: \thanklesspublisher}

\par\smallcaps{github.com/youth-code-academy/book-programming-science}

\par This work is licensed under the Creative Commons Attribution-NonCommercial-ShareAlike
4.0 International License. To view a copy of this license, 
visit \url{http://creativecommons.org/licenses/by-nc-sa/4.0/} or send a letter to
Creative Commons,
PO Box 1866, Mountain View, CA 94042, USA.\index{license}

\par\textit{First printing, \monthyear}
\end{fullwidth}

% r.5 contents
\tableofcontents

\listoffigures

\listoftables

% r.7 dedication
\cleardoublepage
~\vfill
\begin{doublespace}
\noindent\fontsize{18}{22}\selectfont\itshape
\nohyphenation
Dedicated to all the young students I had the privilege to tutor
so far and allowed me to develop this work.
\end{doublespace}
\vfill
\vfill


\cleardoublepage

% Start the main matter (normal chapters)
\mainmatter

%%%%%%%%%%%%%%%%%%%%
%%%%%%%%%%%%%%%%%%%%
%%
%% Andrea Tino - 2019
%% Programming + Science
%% Intro
%%
%%%%%%%%%%%%%%%%%%%%
%%%%%%%%%%%%%%%%%%%%

\section{Introduction}
\label{sec:intro}

We focus on a specific linguistic problem concerning American Sign Language (ASL) and the possibility
of creating a system (hardware + software) able to read and translate into English the corresponding sign
flow. Although literature is already full of such applications, this effort is characterized by a different context, 
determined
by a set of peculiar conditions causing the challenge not to be addressed before. 
The objective is to allow the correct
interpretation of the sign flow generated by a Deaf or Hard of Hearing (D/HH) subject, by only relying on 
data coming from
one device, placed on the subject's dominant hand; this will limit the parsing software to practically being able
to read (almost) only half of the signs, determining a lack of information to be coped with in order to
reconstruct the original flow.

Though the scope is specific to one type of Sign Language (ASL) 
and to one output language (English) in the translation
process, the aim is to ultimately find results which can be generic and applicable to all scenarios. Therefore,
the bigger aim is to devise equations valid for every 
\textit{ideogrammatic} language\footnote{As we regard languages according to the well-known 
models in Computational Linguistics, an ideogrammatic language is intended to be one whose
writing system (tokens) is based on ideograms.} whose tokens (lexems) consist
of two symbols (in Sign Languages, one symbol for the left hand, and one for the right). 
The main hypothesis is to consider one of the symbols in every pair in each lexem as
dominant, and the other one as non-dominant, then modelling one flow of signs as a combination of two
parallel flows, and subsequently remove the information about the non-dominant
flow; the challenge is being able to reconstruct the original string of tokens.


%%%%%%%%%%%%%%%%%%%%
%%%%%%%%%%%%%%%%%%%%
%%
%% Andrea Tino - 2019
%% Programming + Science
%% Opinion model
%%
%%%%%%%%%%%%%%%%%%%%
%%%%%%%%%%%%%%%%%%%%

\section{Building our first CA: Conway's Game of Life}
\label{sec:simpleca}

We want to create our first CA in code so that we can display the cells and see them changing
state. In this section, we will build the basic architecture of a CA that can be used to build any
CA in future. For this, we are going to use the latest web technologies to create web sites and
web applications in the browser: Javascript, HTML and CSS.

\subsection{Creating the basic project structure}
In our PC, let's create a directory (anywhere you want, on your Desktop maybe?) and give it
a cool name like: \texttt{cellautom}. Inside this new directory, do the following:

\begin{enumerate}
\item Create a file and name it: \texttt{index.html}.
\item Create a file and name it: \texttt{ca.js}.
\item Create one last file and name it: \texttt{style.css}.
\end{enumerate}

These three files represent the basic organization of our visual CA we are going to develop.
Table \ref{tab:files} offers a good overview of what they are needed for.

%
% Table
%
\begin{table}[!t]
\centering
\caption{Please write your table caption here}
\label{tab:files}
%
% Follow this input for your own table layout
%
\begin{tabular}{p{0.2\textwidth}p{0.2\textwidth}p{0.5\textwidth}}
\hline\noalign{\smallskip}
File & Type & Description \\
\noalign{\smallskip}\svhline\noalign{\smallskip}
\texttt{index.html} & Web page & This is the web pae that will display the CA and its evolution.\\
\texttt{ca.js} & Javascript code file  & This file will contain the J
    avascript code that will make the CA appear and evolve.\\
\texttt{style.css} & CSS Stylesheet  & The stylesheet we will use to apply colors, fonts 
    and make our CA beautiful.\\
\noalign{\smallskip}\hline\noalign{\smallskip}
\end{tabular}
\end{table}
%

% Figure
%
\begin{figure}[b]
\sidecaption
% tikz diagram
\input{diag-dirstruct1}
%
% If not, use
%\picplace{5cm}{2cm} % Give the correct figure height and width in cm
%
\caption{Your project folder should look like this.}
\label{fig:dirstruct1}
\end{figure}
%

\subsection{Defining the barebones}
Time to write some initial code to see something appearing on the page once we run in the browser.
Open file \texttt{index.html} with tyour favorite editor and input this code:

\begin{programcode}{index.html}
Write this code minding casing and spacing.
\begin{verbatim}
<!DOCTYPE html>
<html>
<head>
  <title>My Cellular Automaton</title>
</head>

<body>
  Hello world!
</body>
</html>
\end{verbatim}
\end{programcode}

Save the file and now try to open it in your browser.

\begin{tips}{A first glance at HTML}
The code we just wrote is read by the browser to create a graphical visualization. HTML is used to
create web pages. It is not a \textit{programming language} (it does not tell a computer what to do),
but a \textit{markup language} (it tells a computer what to display and paint on the screen).

A minimal HTML page is exactly the one we wrote. It is all based on the concept of \textit{tags}. The
first line \texttt{<!DOCTYPE html>} is special and tells the browser that we are using the latest
version of HTML (you should always use this). Then a new tag \texttt{<html>} is opened and is
closed at the end of the file: \texttt{</html>}. An opening tag and a closing tag make a \textit{block}.
Blocks can contain other blocks.

The \texttt{<html>} block must contain, in order, two other blocks:
\texttt{<head>} and \texttt{<body>}. The first block contains the block for defining the title of the page
(this text is displayed on the browser's top bar). Everything inside \texttt{<head>} will not generate any 
graphics, it only contains information about the page. What's inside \texttt{body} is, on the other hand, 
painted (or, more technically speaking, rendered\footnote{The term \textit{render} is used to indicate
the complex set of operations that a program does in order to visualize something on the screen.})
inside the browser window. As you can see, we only have a piece of 
text\footnote{The \textit{Hello World} is, historically, the first thing one learns
to do when learning a new programming language, we had to respect tradition here.}, which
is in fact rendered on a blank, empty page.
\end{tips}

Of course we don't want to just display text, we want to render a full CA! So, in the same file, replace
that text.

\begin{programcode}{index.html (snippet)}
Remove \texttt{Hello world!} and insert a \texttt{<div>} block instead.
\begin{verbatim}
<body>
  <div id="ca"></div>
</body>
\end{verbatim}
\end{programcode}

A \texttt{<div>} block is used to group things. We are going to write some code that puts some graphics
inside it. Before leaving this file, we need to import inside it the other two files we have created.

\begin{programcode}{index.html (snippet)}
Place these new tags right below block \texttt{<title>}.
\begin{verbatim}
<head>
  <title>My Cellular Automaton</title>
  <script src="ca.js"></script>
  <link rel="stylesheet" href="style.css">
</head>
\end{verbatim}
\end{programcode}

The first new tag we have added is a \texttt{<script>} which instructs the browser to load and run the
Javascript code inside \texttt{ca.js}. The next one is a \texttt{<link>} tag (this one does not have a closing tag)
and tells the browser to load the styles defined inside \texttt{style.css}.

\subsection{Creating the grid}
For now we are done with \texttt{index.html}, so let.

\subsection{Coding the evolution logic}
For now w


%%%%%%%%%%%%%%%%%%%%
%%%%%%%%%%%%%%%%%%%%
%%
%% Andrea Tino - 2019
%% Programming + Science
%% Opinion model
%%
%%%%%%%%%%%%%%%%%%%%
%%%%%%%%%%%%%%%%%%%%

\section{Developing a CA to describe how people change opinion}
\label{sec:opinionca}

Following the typical approaches in Computational Linguistics, we need to formalise the target language.
A valid sign is therefore represented by an ordered 2-dimension tuple:

In the context of our effort, we will not handle strings of full tokens $\Phi \in \mathcal{S}^\ast$ as we will
be missing, for each token $s = (\Sigma, \sigma) \in \mathcal{S}$, the information on $\sigma$.


%
% Andrea Tino
% 2020
%
% =======================================================================
% Creative Commons Attribution-NonCommercial-ShareAlike 4.0 International
% Public License
% =======================================================================
%

\chapter{Final remarks}

We have completed our experience with CA. As you could see, they are beautiful mathematical,
programmable, visual structures that have a variety of different applications. We have experimented
with them by crafting one automaton to study and solve problems in a society; but there are many
more fields where CA can prove to be very useful so we encourage you to keep experimenting more
by creating more automata and check their evolutions.


%%%%%%%%%%%%%%%%%%%%
%%%%%%%%%%%%%%%%%%%%
%%
%% Andrea Tino - 2019
%% Solutions
%
%% Programming + Science
%% Cellular Automata
%%
%%%%%%%%%%%%%%%%%%%%
%%%%%%%%%%%%%%%%%%%%

\Extrachap{Solutions}

\section*{Problems in Chapter~\ref{sec:intro}}

\begin{sol}{prob:evolvetoward111}
\textbf{Find the remaining evolutions}\\
We have already checked, in example \ref{ex:stateled4}, that
$(0,1,0) \overset{f}{\rightarrow} (1,1,1)$, so we need to check the other possible
states different from $(0,0,0)$.
\begin{itemize}
\item If we consider $x_0 = (1,0,0)$, then we have that: $(1,0,0) \rightarrow (1,0,1) \rightarrow (1,1,1)$.
\item If we consider $x_0 = (0,0,1)$, then we have that: $(0,0,1) \rightarrow (0,1,1) \rightarrow (1,1,1)$.
\end{itemize}
The remaining initial configurations are: $(1,1,0)$, $(1,0,1)$, $(0,1,1)$ and $(1,1,1)$. But those are already
known because they are covered by the evolution of the initial configurations we covered before. For example,
$(1,1,0)$ is the second configuration the CA assumes after $(0,1,0)$.
\end{sol}


\begin{sol}{prob:testcyclic}
\textbf{Period of a cyclic initial condition}\\
In example \ref{ex:stateled3}, we need to start from $x_0 = (1,0,1)$ and calculate the evolution of the automaton.
We have that $x_1 = (0,1,1)$, then $x_2 = (1,1,0)$ and $x_3 = x_ 0 = (1,0,1)$. We got back to the initial
condition after $T=3$ cycles.
\end{sol}

\begin{sol}{prob:changecasize}
\textbf{Change the size of the CA}\\
In \texttt{ca.js}, inside our Javascript module, we have created two contants to use to set the size of the
automaton. If we change their values, we will get a different size. For example, to have a 20x20 grid,
we just do:
\begin{code}
const rowsnum = 20;
const colsnum = 20;
\end{code}
Or:
\begin{code}
const rowsnum = 20;
const colsnum = 30;
\end{code}
If we want a rectangular automaton.
\end{sol}

\begin{sol}{prob:changeinit}
\textbf{Setting a different initial condition}\\
We have created, inside our Javascript module in \texttt{ca.js}, a constant called: \texttt{initConfig}.
By specifying different cell IDs, we can set a different initial configuration:
\begin{code}
const initConfig = ["3:1", "1:3", "4:4"];
\end{code}
\end{sol}

\begin{sol}{prob:cacolor}
\textbf{Active cells with a different color}\\
We set to black the color of active cells. If you remember, we set this color in our stylesheet
\texttt{style.css}. In there, we need to change the CSS rule for active cells and change the value
of CSS property \texttt{background-color}:
\begin{codeh1}{1}{3}
.cell[data-value="1"] {
  background-color: #f00;
}
\end{codeh1}
In this example, we set it to red.
\end{sol}

\begin{sol}{prob:flowers}
\textbf{Use an image for active cells}\\
To use an image instead of a color to signal that a cell is active, we need to find an image and
make it available in our project:
\begin{enumerate}
  \item In your project folder, create another folder and call it: \texttt{images}.
  \item Search in the Internet for an image you like. Try to find a PNG because they will probably have
    a transparent background. If you don't want to spend time doing it, we have an image for you:
    \texttt{flower.png} which you can find under folder \texttt{v2.4.1/images}.
  \item Move the image inside the folder you created in the first step.
\end{enumerate}
Now we need to change the style
of active cells, which means we must edit \texttt{style.css}:
\begin{codeh1}{1}{5}
.cell[data-value="1"] {
  background-image: url("images/flower.png");
  background-position: center;
  background-size: cover;
}
\end{codeh1}
With property \texttt{background-image} we reference the image we want to use (the path is relative to the
CSS file location). Since we want to make sure that the image is placed at the center of the cell, we must use
\texttt{background-position: center}. Finally, we want to ensure that the image exactly takes the size of the
cell and adapts to it, therefore we use \texttt{background-size: cover} which accomplishes exactly that.

If you go ahead and try it (save the stylesheet and refresh the page),
you will see your image popping up where active cells are! Also, if you change the size of a cell,
by changing the value of constant \texttt{cellsize} inside our Javascript module in \texttt{ca.js},
you will see the image will adapt every time.
\end{sol}

\begin{sol}{prob:nextasauto}
\textbf{Automatic evolution of the automaton}\\
Let's start by creating the interface controls: we need to relabel button "Next", and create button "Stop".
In \texttt{index.html}, let's make these modifications:
\begin{codehtmlh1}{1}{4}
<div class="controls">
  <button id="buttonStart">Start</button>
  <button id="buttonStop">Stop</button>
  <span id="cycleText"></span>
</div>
\end{codehtmlh1}
Note that we have both changed the text inside button "Next", as well as changed its ID.
We can now move to \texttt{style.css} to make sure that the styling we had on button "Next" is now
applied to both buttons:
\begin{codecss}
#buttonStart, #buttonStop {
  background-color: #ddd;
  color: #000;
  border: none;
  padding: 5px;
}
#buttonStart:hover, #buttonStop:hover {
  background-color: #ccc;
}
#buttonStart:active, #buttonStop:active {
  background-color: #bbb;
}
\end{codecss}
We have just changed the selectors of the last 3 CSS rules in the file.

The next step is now ading the logic to the buttons, therefore we need to make some changes inside
\texttt{ca.js}. Our strategy is to basically execute the code inside the event handler for button "Next" in
function \texttt{initializeButton} every 1 second; to do that, we need to set an
\texttt{interval}\footnote{In Javascript, as \texttt{interval} is a function that is registered to be
executed every $n \in \mathbb{Q}$ seconds.}. So, let's create a variable at the beginning of the module to
host the interval ID (we will need it to cancel the interval when clicking on button "Stop"):
\begin{codeh1}{1}{3}
let t = 0; // Cycles (time)
let int = 0; // CA auto update interval
\end{codeh1}
We then move to function \texttt{initializeButton} and rename it into: \texttt{initializeButtons} as it will now
set the click event listeners to both buttons, not just one. We also modify its logic, so make sure
to redefine the function like this:
\begin{code}
function initializeButtons() {
  // Button "Start"
  let buttonStart = document.getElementById("buttonStart");
  buttonStart.addEventListener("click", function(){
    startAutoUpdate();
  });

  // Button "Stop"
  let buttonStop = document.getElementById("buttonStop");
  buttonStop.addEventListener("click", function(){
    stopAutoUpdate();
  });
  buttonStop.disabled = true;
}
\end{code}
As you can see, we now do sometghing different when the buttons are clicked: we invoke two functions: 
\texttt{startAutoUpdate} and \texttt{stopAutoUpdate} that we still need to write. Also note how we
disable the "Stop" button at the beginning, because, when the page loads, we will only allow the user
to start the evolution.
Since we changed the name of the function, we need to update the place where we invoke it, so down in the module,
make sure to make this modification:
\begin{codeh1}{1}{3}
initializeGrid();
initializeButtons();
updateCycleText();
\end{codeh1}
Time to code the two missing functions we mentioned before, add them below function \texttt{initializeButtons}:
\begin{code}
function startAutoUpdate() {
  int = window.setInterval(function(){
    next();
    updateCycleText();
  }, 1000);

  // Disable the "Start" button
  let buttonStart = document.getElementById("buttonStart");
  buttonStart.disabled = true;
  // Enable the "Stop" button
  let buttonStop = document.getElementById("buttonStop");
  buttonStop.disabled = false;
}

function stopAutoUpdate() {
  window.clearInterval(int);

  // Enable the "Start" button
  let buttonStart = document.getElementById("buttonStart");
  buttonStart.disabled = false;
  // Disable the "Stop" button
  let buttonStop = document.getElementById("buttonStop");
  buttonStop.disabled = true;
}
\end{code}
The first function is executed when button "Start" is pressed, it creates the interval which, after every 1 second,
calls \texttt{next} and \texttt{updateCycleText} (they were called before, once, when clicking on button "Next").
The second function stops the interval. Also note how we handle the enabled/disabled states of the buttons to make
sure that, once the auto-update is started, the user cannot start a new one, as well as ensuring that the user
cannot click stop when the evolution has already been interrupted before and never resumed.
\end{sol}

\begin{sol}{prob:blackcounter}
\textbf{Counting the number of black cells}\\
Let's start from \texttt{index.html} in order to create the new label:
\begin{codehtmlh1}{3}{5}
<div class="controls">
  <button id="buttonNext">Next</button>
  <span id="cycleText"></span>
  <span id="blackCountText"></span>
</div>
\end{codehtmlh1}
We have added a new label with ID \texttt{blackCountText}. Let's now move to \texttt{ca.js} to create the logic to
update the text inside the label every time the CA changes. We will count the number of black cells by using a
very simple (and slow) approach: every time the automaton advances in a new configuration, we scan all the cells and
count the number of black ones. So, in the module, add this function after function \texttt{updateCycleText}:
\begin{code}
function updateBlackCountText() {
  let count = 0;
  for (let i = 1; i <= rowsnum; i++) {
    for (let j = 1; j <= colsnum; j++) {
      if (get(i, j) === 1) {
          count++;
      }
    }
  }

  let text = document.getElementById("blackCountText");
  text.textContent = "black cells: " + count;
}
\end{code}
As you can see, we go through every cell (border included), and increment variable \texttt{counter} every time
a black cell is encountered. The final part of the function will update the text of the label we added to the
page at the beginning.
Next, we need to call this function, basically in every place where we also call function \texttt{updateCycleText};
that is when the CA is rendered (at the end of the module):
\begin{codeh1}{2}{4}
initializeButton();
updateCycleText();
updateBlackCountText();
\end{codeh1}
And every time the automaton changes (inside function \texttt{initializeButton}):
\begin{codeh1}{2}{4}
next();
updateCycleText();
updateBlackCountText();
\end{codeh1}
As mentioned, this approach makes the application slower because we have to re-scan again all the cells.
You can try to find a better approach!
\end{sol}

\begin{sol}{prob:cgl1}
\textbf{Blocks in Conway's Game of Life}\\
We can use the original size (9x9) and start from cell \texttt{3:3}. We must create
an initial condition with two consecutive active cells, and two active cells below them:
\begin{code}
const initConfig = ["3:3", "3:4", "4:3", "4:4"];
\end{code}
When we save and refresh the page, a block appears. If we click \textit{Next} and follow the
evolution, we see that nothing changes. So a \textit{block}, is a static condition
(see definition \ref{def:staticconf}) in CGL.
\end{sol}

\begin{sol}{prob:cgl2}
\textbf{Bee-hives and tubs in Conway's Game of Life}\\
Let's start creating a \textit{bee-hive} first as shown in figure \ref{fig:cglplay}. We can use
the original 9x9 automaton size because this configuration's size is 3x4, so it fits.
We place the first (left-most and top-most) black cell of the configuration in cell \texttt{3:3},
so we write:
\begin{code}
const initConfig = ["3:3", "3:4", "4:2", "4:5", "5:3", "5:4"];
\end{code}
When we refresh the page we see the configuration and as we advance the CA, we can see that
nothing changes. So a bee-hive is a static configuration.

Let's now draw a \textit{tub} as static configuration.
This initial configuration's size is 3x3 so it fits the original 9x9 CA.
Its first black cell will be placed in position \texttt{3:3}, so we can write:
\begin{code}
const initConfig = ["3:3", "4:2", "4:4", "5:3"];
\end{code}
Let's refresh and make the automaton evolve: again another static configuration.

If we want to draw both a bee-hive and a tub, we need a bigger automaton. The bee-hive
is 3x4 and the tub is 3x3, we also need to leave at least 3 cell separation
(to avoid one figure to affect the other) between the two figures, which means that a
15x15 automaton should be ok. We will draw a bee-hive on the top-left part of the automaton
and a tub in the bottom-right portion:
\begin{code}
const initConfig = ["3:3", "3:4", "4:2", "4:5", "5:3", "5:4", "10:10", "11:9", "11:11", "12:10"];
\end{code}
As we refresh and make the automaton evolve, we see that the two figures remain there, so a
configuration obtained by combining two static configurations is still static.

Let's have some more fun and experiment a bit longer. What happens if the two static figures
are placed too close to each other? If we try to move the tub very close to the bee-hive like this:
\begin{code}
const initConfig = ["3:3", "3:4", "4:2", "4:5", "5:3", "5:4", "5:6", "6:5", "6:7", "7:6"];
\end{code}
We can see, as we refresh the page and let the CA evolve, that the two figures start changing.
That's because each cell reacts to a neighborhood of radius 1, since one figure's cells are
in the other's cells' neighborhoods, a different evolution happens. In this case the initial configuration
we just created leads to a final configuration which is a bee-hive.
\end{sol}

\begin{sol}{prob:cgl3}
\textbf{Blinkers in Conway's Game of Life}\\
In the original automaton (9x9), we can draw an horizontal \textit{blinker}:
\begin{code}
const initConfig = ["5:5", "5:6", "5:7"];
\end{code}
Or a vertical one:
\begin{code}
const initConfig = ["5:5", "6:5", "7:5"];
\end{code}
In both cases, as we make the automaton evolve, we see that the two initial configurations are
cyclic because we get back to them after two cycles (period $T=2$).
\end{sol}

\begin{sol}{prob:cgl4}
\textbf{Toads and beacons in Conway's Game of Life}\\
Let's first draw a \textit{toad} as shown in figure \ref{fig:cglplay};
this figure takes a 2x4 rectangle, so our original automaton (9x9)
is ok to contain it. We set the first black cell to be \texttt{5:5}:
\begin{code}
const initConfig = ["5:5", "5:6", "5:7", "6:4", "6:5", "6:6"];
\end{code}
As we make this CA evolve, we see a cyclic configuration with period $T=2$
(as it repeats after two cycles).

As for the \textit{beacon}, figure \ref{fig:cglplay} provides us its size: 4x4, which fits
our original CA, so we don't need to change its size. If we set \texttt{3:3} to be the first black
cell of this configuration, we have:
\begin{code}
const initConfig = ["3:3", "3:4", "4:3", "4:4", "5:5", "5:6", "6:5", "6:6"];
\end{code}
Again, trying the evolution, we can see that this is also a recurrent configiration with period $T=2$.
\end{sol}

\begin{sol}{prob:cgl5}
\textbf{Gliders in Conway's Game of Life}\\
Figure \ref{fig:cglplay} shows how a \textit{glider} should look like.  We take
the first black cell of this confgiguration to be placed in cell \texttt{3:3}:
\begin{codeh2}{0}{3}{5}{7}
const rowsnum = 30;
const colsnum = 30;
const cellsize = 20; // In px
const initConfig = ["3:3", "4:4", "4:5", "5:3", "5:4"];
\end{codeh2}
As you can see, we also changed the size to 30x30 as requested by the problem.
When we try to make the automaton evolve, we see that the figure sorts of moves towards the bottom-right
part of the grid. Although the shape we see repeats itself, its position changes every time,
so we cannot say this initial configuration is cyclic. This is a configuration where the automaton does
not evolve into a specific final configuration, but keeps changing indefinitely (if it were infinite in size).
In our case, however, because of the border effect, the automaton will eventually turn into a block at cycle
$t=100$ and stay there. So, in our case, the glider condition will lead into a final configuration which is
a block.
\end{sol}

\begin{sol}{prob:cgl6}
\textbf{Spaceships in Conway's Game of Life}\\
In figure \ref{fig:cglplay}, we can see the \textit{spaceship} configuration is the biggest listed.
We set the first black cell of this initial condition to be in cell \texttt{7:7}:
\begin{codeh2}{0}{3}{5}{7}
const rowsnum = 50;
const colsnum = 50;
const cellsize = 20; // In px
const initConfig = ["7:7", "7:8", "8:5", "8:6", "8:8", "8:9", "9:5", "9:6", "9:7", "9:8", "10:6", "10:7"];
\end{codeh2}
We have also changed the automaton's size to 50x50.
As we make the CA evolve, we see that the behavior is the same as in solution \ref{prob:cgl5}: the shape
moves down until reaching the border. If the CA had an infinite size, the shape would move continuously
downward, and the CA would never reach a final configuration, thus making this initial condition, a
\textit{divergent} one as explained in definition \ref{def:divconf}.
\end{sol}

\begin{sol}{prob:blocksizedie}
\textbf{Vanishing rectangular blocks}\\
In fi
\end{sol}

\begin{sol}{prob:highernblockslive}
\textbf{Growing isolated blocks}\\
In fi
\end{sol}

\begin{sol}{prob:opinionproof1}
\textbf{Alternative block connections}\\
In fi
\end{sol}





%%
% The back matter contains appendices, bibliographies, indices, glossaries, etc.

\backmatter

\bibliography{bibliography}
\bibliographystyle{plainnat}

%%
\printindex

\end{document}

