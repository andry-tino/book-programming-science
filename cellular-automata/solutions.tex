%%%%%%%%%%%%%%%%%%%%
%%%%%%%%%%%%%%%%%%%%
%%
%% Andrea Tino - 2019
%% Solutions
%
%% Programming + Science
%% Cellular Automata
%%
%%%%%%%%%%%%%%%%%%%%
%%%%%%%%%%%%%%%%%%%%

\Extrachap{Solutions}

\section*{Problems in Chapter~\ref{sec:intro}}

\begin{sol}{prob:evolvetoward111}
\textbf{Find the remaining evolutions}\\
We have already checked, in example \ref{ex:stateled4}, that
$(0,1,0) \overset{f}{\rightarrow} (1,1,1)$, so we need to check the other possible
states different from $(0,0,0)$.
\begin{itemize}
\item If we consider $x_0 = (1,0,0)$, then we have that: $(1,0,0) \rightarrow (1,0,1) \rightarrow (1,1,1)$.
\item If we consider $x_0 = (0,0,1)$, then we have that: $(0,0,1) \rightarrow (0,1,1) \rightarrow (1,1,1)$.
\end{itemize}
The remaining initial configurations are: $(1,1,0)$, $(1,0,1)$, $(0,1,1)$ and $(1,1,1)$. But those are already
known because they are covered by the evolution of the initial configurations we covered before. For example,
$(1,1,0)$ is the second configuration the CA assumes after $(0,1,0)$.
\end{sol}


\begin{sol}{prob:testcyclic}
\textbf{Period of a cyclic initial condition}\\
In example \ref{ex:stateled3}, we need to start from $x_0 = (1,0,1)$ and calculate the evolution of the automaton.
We have that $x_1 = (0,1,1)$, then $x_2 = (1,1,0)$ and $x_3 = x_ 0 = (1,0,1)$. We got back to the initial
condition after $T=3$ cycles.
\end{sol}

\begin{sol}{prob:changecasize}
\textbf{Change the size of the CA}\\
In \texttt{ca.js}, inside our Javascript module, we have created two contants to use to set the size of the
automaton. If we change their values, we will get a different size. For example, to have a 20x20 grid,
we just do:
\begin{code}
const rowsnum = 20;
const colsnum = 20;
\end{code}
Or:
\begin{code}
const rowsnum = 20;
const colsnum = 30;
\end{code}
If we want a rectangular automaton.
\end{sol}

\begin{sol}{prob:changeinit}
\textbf{Setting a different initial condition}\\
We have created, inside our Javascript module in \texttt{ca.js}, a constant called: \texttt{initConfig}.
By specifying different cell IDs, we can set a different initial configuration:
\begin{code}
const initConfig = ["3:1", "1:3", "4:4"];
\end{code}
\end{sol}

\begin{sol}{prob:cacolor}
\textbf{Active cells with a different color}\\
We set to black the color of active cells. If you remember, we set this color in our stylesheet
\texttt{style.css}. In there, we need to change the CSS rule for active cells and change the value
of CSS property \texttt{background-color}:
\begin{codeh1}{1}{3}
.cell[data-value="1"] {
  background-color: #f00;
}
\end{codeh1}
In this example, we set it to red.
\end{sol}

\begin{sol}{prob:flowers}
\textbf{Use an image for active cells}\\
Todo
\end{sol}

\begin{sol}{prob:cgl1}
\textbf{Blocks in Conway's Game of Life}\\
We can use the original size (9x9) and start from cell \texttt{3:3}. We must create
an initial condition with two consecutive active cells, and two active cells below them:
\begin{code}
const initConfig = ["3:3", "3:4", "4:3", "4:4"];
\end{code}
When we save and refresh the page, a block appears. If we click \textit{Next} and follow the
evolution, we see that nothing changes. So a \textit{block}, is a static condition
(see definition \ref{def:staticconf}) in CGL.
\end{sol}

\begin{sol}{prob:cgl2}
\textbf{Bee-hives and tubs in Conway's Game of Life}\\
Let's start creating a \textit{bee-hive} first as shown in figure \ref{diag-cglplay}. We can use
the original 9x9 automaton size because this configuration's size is 3x4, so it fits.
We place the first (left-most and top-most) black cell of the configuration in cell \texttt{3:3},
so we write:
\begin{code}
const initConfig = ["3:3", "3:4", "4:2", "4:5", "5:3", "5:4"];
\end{code}
When we refresh the page we see the configuration and as we advance the CA, we can see that
nothing changes. So a bee-hive is a static configuration.

Let's now draw a \textit{tub} as static configuration.
This initial configuration's size is 3x3 so it fits the original 9x9 CA.
Its first black cell will be placed in position \texttt{3:3}, so we can write:
\begin{code}
const initConfig = ["3:3", "4:2", "4:4", "5:3"];
\end{code}
Let's refresh and make the automaton evolve: again another static configuration.

If we want to draw both a bee-hive and a tub, we need a bigger automaton. The bee-hive
is 3x4 and the tub is 3x3, we also need to leave at least 3 cell separation
(to avoid one figure to affect the other) between the two figures, which means that a
15x15 automaton should be ok. We will draw a bee-hive on the top-left part of the automaton
and a tub in the bottom-right portion.
\end{sol}

\begin{sol}{prob:cgl3}
\textbf{Blinkers in Conway's Game of Life}\\
In examp
\end{sol}

\begin{sol}{prob:cgl4}
\textbf{Toads and beacons in Conway's Game of Life}\\
In examp
\end{sol}

\begin{sol}{prob:cgl5}
\textbf{Gliders in Conway's Game of Life}\\
In examp
\end{sol}

\begin{sol}{prob:cgl6}
\textbf{Spaceships in Conway's Game of Life}\\
In examp
\end{sol}
