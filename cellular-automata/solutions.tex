%%%%%%%%%%%%%%%%%%%%
%%%%%%%%%%%%%%%%%%%%
%%
%% Andrea Tino - 2019
%% Solutions
%
%% Programming + Science
%% Cellular Automata
%%
%%%%%%%%%%%%%%%%%%%%
%%%%%%%%%%%%%%%%%%%%

\Extrachap{Solutions}

\section*{Problems in Chapter~\ref{sec:intro}}

\begin{sol}{prob:evolvetoward111}
\textbf{Find the remaining evolutions}\\
We have already checked, in example \ref{ex:stateled4}, that
$(0,1,0) \overset{f}{\rightarrow} (1,1,1)$, so we need to check the other possible
states different from $(0,0,0)$.
\begin{itemize}
\item If we consider $x_0 = (1,0,0)$, then we have that: $(1,0,0) \rightarrow (1,0,1) \rightarrow (1,1,1)$.
\item If we consider $x_0 = (0,0,1)$, then we have that: $(0,0,1) \rightarrow (0,1,1) \rightarrow (1,1,1)$.
\end{itemize}
The remaining initial configurations are: $(1,1,0)$, $(1,0,1)$, $(0,1,1)$ and $(1,1,1)$. But those are already
known because they are covered by the evolution of the initial configurations we covered before. For example,
$(1,1,0)$ is the second configuration the CA assumes after $(0,1,0)$.
\end{sol}


\begin{sol}{prob:testcyclic}
\textbf{Period of a cyclic initial condition}\\
In example \ref{ex:stateled3}, we need to start from $x_0 = (1,0,1)$ and calculate the evolution of the automaton.
We have that $x_1 = (0,1,1)$, then $x_2 = (1,1,0)$ and $x_3 = x_ 0 = (1,0,1)$. We got back to the initial
condition after $T=3$ cycles.
\end{sol}

\begin{sol}{prob:changecasize}
\textbf{Change the size of the CA}\\
In \texttt{ca.js}, inside our Javascript module, we have created two contants to use to set the size of the
automaton. If we change their values, we will get a different size. For example, to have a 20x20 grid,
we just do:
\begin{code}
const rowsnum = 20;
const colsnum = 20;
\end{code}
Or:
\begin{code}
const rowsnum = 20;
const colsnum = 30;
\end{code}
If we want a rectangular automaton.
\end{sol}

\begin{sol}{prob:changeinit}
\textbf{Setting a different initial condition}\\
We have created, inside our Javascript module in \texttt{ca.js}, a constant called: \texttt{initConfig}.
By specifying different cell IDs, we can set a different initial configuration:
\begin{code}
const initConfig = ["3:1", "1:3", "4:4"];
\end{code}
\end{sol}

\begin{sol}{prob:cacolor}
\textbf{Active cells with a different color}\\
We set to black the color of active cells. If you remember, we set this color in our stylesheet
\texttt{style.css}. In there, we need to change the CSS rule for active cells and change the value
of CSS property \texttt{background-color}:
\begin{codeh1}{1}{3}
.cell[data-value="1"] {
  background-color: #f00;
}
\end{codeh1}
In this example, we set it to red.
\end{sol}

\begin{sol}{prob:flowers}
\textbf{Use an image for active cells}\\
To use an image instead of a color to signal that a cell is active, we need to find an image and
make it available in our project:
\begin{enumerate}
  \item In your project folder, create another folder and call it: \texttt{images}.
  \item Search in the Internet for an image you like. Try to find a PNG because they will probably have
    a transparent background. If you don't want to spend time doing it, we have an image for you:
    \texttt{flower.png} which you can find under folder \texttt{v2.4.1/images}.
  \item Move the image inside the folder you created in the first step.
\end{enumerate}
Now we need to change the style
of active cells, which means we must edit \texttt{style.css}:
\begin{codeh1}{1}{5}
.cell[data-value="1"] {
  background-image: url("images/flower.png");
  background-position: center;
  background-size: cover;
}
\end{codeh1}
With property \texttt{background-image} we reference the image we want to use (the path is relative to the
CSS file location). Since we want to make sure that the image is placed at the center of the cell, we must use
\texttt{background-position: center}. Finally, we want to ensure that the image exactly takes the size of the
cell and adapts to it, therefore we use \texttt{background-size: cover} which accomplishes exactly that.

If you go ahead and try it (save the stylesheet and refresh the page),
you will see your image popping up where active cells are! Also, if you change the size of a cell,
by changing the value of constant \texttt{cellsize} inside our Javascript module in \texttt{ca.js},
you will see the image will adapt every time.
\end{sol}

\begin{sol}{prob:cgl1}
\textbf{Blocks in Conway's Game of Life}\\
We can use the original size (9x9) and start from cell \texttt{3:3}. We must create
an initial condition with two consecutive active cells, and two active cells below them:
\begin{code}
const initConfig = ["3:3", "3:4", "4:3", "4:4"];
\end{code}
When we save and refresh the page, a block appears. If we click \textit{Next} and follow the
evolution, we see that nothing changes. So a \textit{block}, is a static condition
(see definition \ref{def:staticconf}) in CGL.
\end{sol}

\begin{sol}{prob:cgl2}
\textbf{Bee-hives and tubs in Conway's Game of Life}\\
Let's start creating a \textit{bee-hive} first as shown in figure \ref{fig:cglplay}. We can use
the original 9x9 automaton size because this configuration's size is 3x4, so it fits.
We place the first (left-most and top-most) black cell of the configuration in cell \texttt{3:3},
so we write:
\begin{code}
const initConfig = ["3:3", "3:4", "4:2", "4:5", "5:3", "5:4"];
\end{code}
When we refresh the page we see the configuration and as we advance the CA, we can see that
nothing changes. So a bee-hive is a static configuration.

Let's now draw a \textit{tub} as static configuration.
This initial configuration's size is 3x3 so it fits the original 9x9 CA.
Its first black cell will be placed in position \texttt{3:3}, so we can write:
\begin{code}
const initConfig = ["3:3", "4:2", "4:4", "5:3"];
\end{code}
Let's refresh and make the automaton evolve: again another static configuration.

If we want to draw both a bee-hive and a tub, we need a bigger automaton. The bee-hive
is 3x4 and the tub is 3x3, we also need to leave at least 3 cell separation
(to avoid one figure to affect the other) between the two figures, which means that a
15x15 automaton should be ok. We will draw a bee-hive on the top-left part of the automaton
and a tub in the bottom-right portion:
\begin{code}
const initConfig = ["3:3", "3:4", "4:2", "4:5", "5:3", "5:4", "10:10", "11:9", "11:11", "12:10"];
\end{code}
As we refresh and make the automaton evolve, we see that the two figures remain there, so a
configuration obtained by combining two static configurations is still static.

Let's have some more fun and experiment a bit longer. What happens if the two static figures
are placed too close to each other? If we try to move the tub very close to the bee-hive like this:
\begin{code}
const initConfig = ["3:3", "3:4", "4:2", "4:5", "5:3", "5:4", "5:6", "6:5", "6:7", "7:6"];
\end{code}
We can see, as we refresh the page and let the CA evolve, that the two figures start changing.
That's because each cell reacts to a neighborhood of radius 1, since one figure's cells are
in the other's cells' neighborhoods, a different evolution happens. In this case the initial configuration
we just created leads to a final configuration which is a bee-hive.
\end{sol}

\begin{sol}{prob:cgl3}
\textbf{Blinkers in Conway's Game of Life}\\
In the original automaton (9x9), we can draw an horizontal \textit{blinker}:
\begin{code}
const initConfig = ["5:5", "5:6", "5:7"];
\end{code}
Or a vertical one:
\begin{code}
const initConfig = ["5:5", "6:5", "7:5"];
\end{code}
In both cases, as we make the automaton evolve, we see that the two initial configurations are
cyclic because we get back to them after two cycles (period $T=2$).
\end{sol}

\begin{sol}{prob:cgl4}
\textbf{Toads and beacons in Conway's Game of Life}\\
Let's first draw a \textit{toad} as shown in figure \ref{fig:cglplay};
this figure takes a 2x4 rectangle, so our original automaton (9x9)
is ok to contain it. We set the first black cell to be \texttt{5:5}:
\begin{code}
const initConfig = ["5:5", "5:6", "5:7", "6:4", "6:5", "6:6"];
\end{code}
As we make this CA evolve, we see a cyclic configuration with period $T=2$
(as it repeats after two cycles).

As for the \textit{beacon}, figure \ref{fig:cglplay} provides us its size: 4x4, which fits
our original CA, so we don't need to change its size. If we set \texttt{3:3} to be the first black
cell of this configuration, we have:
\begin{code}
const initConfig = ["3:3", "3:4", "4:3", "4:4", "5:5", "5:6", "6:5", "6:6"];
\end{code}
Again, trying the evolution, we can see that this is also a recurrent configiration with period $T=2$.
\end{sol}

\begin{sol}{prob:cgl5}
\textbf{Gliders in Conway's Game of Life}\\
Figure \ref{fig:cglplay} shows how a \textit{glider} should look like.  We take
the first black cell of this confgiguration to be placed in cell \texttt{3:3}:
\begin{codeh2}{0}{3}{5}{7}
const rowsnum = 30;
const colsnum = 30;
const cellsize = 20; // In px
const initConfig = ["3:3", "4:4", "4:5", "5:3", "5:4"];
\end{codeh2}
As you can see, we also changed the size to 30x30 as requested by the problem.
When we try to make the automaton evolve, we see that the figure sorts of moves towards the bottom-right
part of the grid. Although the shape we see repeats itself, its position changes every time,
so we cannot say this initial configuration is cyclic. This is a configuration where the automaton does
not evolve into a specific final configuration, but keeps changing indefinitely (if it were infinite in size).
In our case, however, because of the border effect, the automaton will eventually turn into a block at cycle
$t=100$ and stay there. So, in our case, the glider condition will lead into a final configuration which is
a block.
\end{sol}

\begin{sol}{prob:cgl6}
\textbf{Spaceships in Conway's Game of Life}\\
In figure \ref{fig:cglplay}, we can see the \textit{spaceship} configuration is the biggest listed.
We set the first black cell of this initial condition to be in cell \texttt{7:7}:
\begin{codeh2}{0}{3}{5}{7}
const rowsnum = 50;
const colsnum = 50;
const cellsize = 20; // In px
const initConfig = ["7:7", "7:8", "8:5", "8:6", "8:8", "8:9", "9:5", "9:6", "9:7", "9:8", "10:6", "10:7"];
\end{codeh2}
We have also changed the automaton's size to 50x50.
As we make the CA evolve, we see that the behavior is the same as in solution \ref{prob:cgl5}: the shape
moves down until reaching the border. If the CA had an infinite size, the shape would move continuously
downward, and the CA would never reach a final configuration, thus making this initial condition, a
\textit{divergent} one as explained in definition \ref{def:divconf}.
\end{sol}
