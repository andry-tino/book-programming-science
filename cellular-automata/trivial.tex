%%%%%%%%%%%%%%%%%%%%
%%%%%%%%%%%%%%%%%%%%
%%
%% Andrea Tino - 2019
%% Programming + Science
%% Trivial
%%
%%%%%%%%%%%%%%%%%%%%
%%%%%%%%%%%%%%%%%%%%

\section{Trivial case: lexically-solvable languages}
\label{sec:lexs}

We start with a simple definition:

\begin{definition}[Lexically-solvable languages] \label{def:lexsolv}
A language $L$ with alphabet $\mathcal{S}$ is defined as \textit{lexically-solvable},
and we write: $L \in \mathcal{L}_{\text{XS}}$, if
and only if each full
token $s = (\Sigma, \sigma) \in \mathcal{S}$ can be reconstructed by simply knowing $\Sigma$.
$\mathcal{L}_{\text{XS}}$ represents the set of lexically-solvable languages.
\end{definition}

The collision set of the full token alphabet of a language in definition \ref{def:colls} is the key to
understanding if the language being considered allows the reconstruction application to be simplified as:
$\Psi : S \mapsto \mathcal{S}$ rather than being $\Psi : S^\ast \mapsto \mathcal{S}^\ast$.
