%%%%%%%%%%%%%%%%%%%%
%%%%%%%%%%%%%%%%%%%%
%%
%% Andrea Tino - 2019
%% Programming + Science
%% Trivial
%%
%%%%%%%%%%%%%%%%%%%%
%%%%%%%%%%%%%%%%%%%%

\section{Trivial case: lexically-solvable languages}
\label{sec:lexs}

We start with a simple definition:

\begin{definition}[Lexically-solvable languages] \label{def:lexsolv}
A language $L$ with alphabet $\mathcal{S}$ is defined as \textit{lexically-solvable},
and we write: $L \in \mathcal{L}_{\text{XS}}$, if
and only if each full
token $s = (\Sigma, \sigma) \in \mathcal{S}$ can be reconstructed by simply knowing $\Sigma$.
$\mathcal{L}_{\text{XS}}$ represents the set of lexically-solvable languages.
\end{definition}

A lexically-solbale language implies our
ability to use only one single dominant token $\Sigma \in S$ to find the corresponding $\sigma \in S$
in the original flow. Such an assumption also implies that $\Psi$ can operate
between $S$ and $\mathcal{S}$.

\begin{definition}[Token collision set] \label{def:colls}
Let $\mathcal{S}$ be the full sign alphabet, we introduce the following quantity:
\begin{equation*}
\Gamma(\mathcal{S}) = \left\{ s = (\Sigma, \sigma) \in \mathcal{S} : 
\exists s^\prime = (\Sigma^\prime, \sigma^\prime) \in \Gamma(\mathcal{S}) : 
s \neq s^\prime \wedge \Sigma = \Sigma^\prime \right\} 
\end{equation*}
As the \textit{collision set} of $\mathcal{S}$.
\end{definition}

\begin{lemma}[Collision set size]
If non empty, the collision set contains at least 2 tokens:
\begin{equation*}
\Gamma(\mathcal{S}) \neq \emptyset \iff \left| \Gamma(\mathcal{S}) \right| > 1
\end{equation*}
\begin{proof}
Immediate, to have a collision at least 2 elements are necessary.
\end{proof}
\end{lemma}

The collision set of the full token alphabet of a language in definition \ref{def:colls} is the key to
understanding if the language being considered allows the reconstruction application to be simplified as:
$\Psi : S \mapsto \mathcal{S}$ rather than being $\Psi : S^\ast \mapsto \mathcal{S}^\ast$.
