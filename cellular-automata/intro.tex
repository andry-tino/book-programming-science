%%%%%%%%%%%%%%%%%%%%
%%%%%%%%%%%%%%%%%%%%
%%
%% Andrea Tino - 2019
%% Programming + Science
%% Intro
%%
%%%%%%%%%%%%%%%%%%%%
%%%%%%%%%%%%%%%%%%%%

\section{Introduction - Why and how do we describe populations?}
\label{sec:intro}

A population is a collection of individuals. It does not necessarily need
to be a set of people, it can be animals. It does not even need to include the
same type of elements, populations can be heterogeneous (e.g. a population
of men and women).

Why do we need to describe and analyze these structures? There are many answers
but the most important is certainly: because our societies are organized in populations!
There are populations everywhere: the city where we live, the building where we reside,
the means of transport we use everyday, the roads we fill with our cars, bycicles and
scooters and so forth. We describe things because we want to extract information, knowledge
that we can use later to achieve some level of control. A practical example is traffic
lights: a crossroad is a limited space where a population of cars happen to spend time.
If we gain more information about the populartion of cars entering and exiting an
intersection, we can later create an algorithm to rework the time durations of green vs.
red signals at each joining road to maximize the 
throughput\footnote{The amount of something in unit of time: $\eta = \frac{N}{T}$. 
Throughput is most often a measure of speed.}.

Urban networks is not the only case scenario where population study comes handy. Other
examples include Biology and Virology: how does an infection propagate over a
populations of individuals? Kermack and McHendrick gave an answer to this question by
formulating the now very well-known \textit{SIR} model.