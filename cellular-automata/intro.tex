%%%%%%%%%%%%%%%%%%%%
%%%%%%%%%%%%%%%%%%%%
%%
%% Andrea Tino - 2019
%% Programming + Science
%% Intro
%%
%%%%%%%%%%%%%%%%%%%%
%%%%%%%%%%%%%%%%%%%%

\section{Introduction}
\label{sec:intro}

We focus on a specific linguistic problem concerning American Sign Language (ASL) and the possibility
of creating a system (hardware + software) able to read and translate into English the corresponding sign
flow. Although literature is already full of such applications, this effort is characterized by a different context, 
determined
by a set of peculiar conditions causing the challenge not to be addressed before. 
The objective is to allow the correct
interpretation of the sign flow generated by a Deaf or Hard of Hearing (D/HH) subject, by only relying on 
data coming from
one device, placed on the subject's dominant hand; this will limit the parsing software to practically being able
to read (almost) only half of the signs, determining a lack of information to be coped with in order to
reconstruct the original flow.

Though the scope is specific to one type of Sign Language (ASL) 
and to one output language (English) in the translation
process, the aim is to ultimately find results which can be generic and applicable to all scenarios. Therefore,
the bigger aim is to devise equations valid for every 
\textit{ideogrammatic} language\footnote{As we regard languages according to the well-known 
models in Computational Linguistics, an ideogrammatic language is intended to be one whose
writing system (tokens) is based on ideograms.} whose tokens (lexems) consist
of two symbols (in Sign Languages, one symbol for the left hand, and one for the right). 
The main hypothesis is to consider one of the symbols in every pair in each lexem as
dominant, and the other one as non-dominant, then modelling one flow of signs as a combination of two
parallel flows, and subsequently remove the information about the non-dominant
flow; the challenge is being able to reconstruct the original string of tokens.
